\section{Attestation-based Governance Systems}

Attestations are cryptographically signed statements that make verifiable claims about real-world attributes, relationships, or events. Unlike traditional credentials or certificates that rely on trusted authorities, attestations create a decentralized web of verifiable claims where anyone can attest to facts they have direct knowledge of. This includes confirming expertise in a domain, vouching for contributions to a project, or verifying participation in events.

While on-chain attestation protocols such as the Ethereum Attestation Service (EAS) have existed for a while, computation over attestation graphs in smart contracts was too expensive to leverage in DAO governance or incentive systems. Verifiable off-chain technologies change this by enabling verifiable computation over attestation graphs while keeping outputs checkable on chain (see Section 3). Networks like TrustGraph demonstrate this approach in production with 65+ experts participating in governance through attestation-based trust scores.

Attestation-based governance creates rich, verifiable social graphs that capture real-world attributes, relationships, and the nuanced reality of community contribution and expertise, rather than relying solely on token holdings or simple credential checks.

A smart contract can verify token ownership. It cannot assess whether someone has the expertise to evaluate a technical proposal, the relationships to broker a partnership, or the reputation to be trusted with treasury decisions. Attestations bridge this gap by enabling verifiable claims about these dimensions of stakeholder legitimacy.

\subsection{Architecture of Attestation-Based Governance}

Verifiable Services, a key off-chain computation technology, process attestation graphs through three functional layers:

\subsubsection{Attestation Infrastructure}

The data collection and verification layer is the primary infrastructure consisting of both on-chain attestation protocols and off-chain collection mechanisms:

\begin{itemize}
\item \textbf{On-chain attestation protocols:} Such as Ethereum Attestation Service (EAS), which provide persistent, cryptographically signed attestations that are publicly verifiable.
\item \textbf{Off-chain credential and peer-reviewed verification:} Systems that collect and verify credentials from external sources (academic credentials, GitHub contributions, professional certifications).
\item \textbf{Standardized claim schemas:} Defining what types of attestations exist and how they are structured for the specific governance context.
\item \textbf{Time-bound and revocable attestations:} Implementing expiry dates and revocation mechanisms to ensure attestations remain current and accurate.
\end{itemize}

\subsubsection{Computation Layer}

This is the key processing layer that performs expensive operations off-chain:

\begin{itemize}
\item \textbf{Graph traversal and analysis:} Computing influence scores, detecting communities, finding paths of trust.
\item \textbf{Multi-attribute scoring algorithms:} Weighing different types of attestations and contributions to generate composite legitimacy scores.
\item \textbf{PageRank-style trust propagation:} Calculating global trust scores based on the attestation graph structure.
\item \textbf{Social distance calculations:} Measuring degrees of separation for delegation limits or trust boundaries.
\item \textbf{Verifiable output generation:} Producing merkle roots, signatures, or ZK proofs that on-chain contracts can efficiently verify.
\end{itemize}

\subsubsection{Integration and Execution Layer}

This layer ties attestation insights to actual governance mechanisms:

\begin{itemize}
\item \textbf{Voting weight determination:} Using attestation scores to calculate voting power for proposals.
\item \textbf{Delegation routing:} Finding appropriate delegates based on expertise attestations.
\item \textbf{Access control:} Gating participation in specific decisions to attestation holders.
\item \textbf{Incentive distribution:} Allocating rewards based on verified contributions.
\item \textbf{Proposal routing:} Matching proposals to reviewers with relevant expertise attestations.
\end{itemize}

\subsubsection{Security Model}

The architecture implements multiple security layers:

\begin{itemize}
\item Attestations themselves are cryptographically signed and tamper-proof.
\item Verifiable Service operators stake bonds that can be slashed for incorrect computation.
\item Multiple operators must agree on computation results (e.g., 3-of-5 threshold).
\item On-chain contracts only accept results that meet verification criteria.
\item Time delays allow for disputes before high-stakes actions execute.
\end{itemize}

\subsubsection{Bootstrapping New Communities}

The cold-start problem is a critical challenge for attestation networks. New communities lack the dense web of attestations needed for meaningful trust scores. Implementations must design bootstrapping mechanisms that balance initial accessibility with long-term security.

One approach uses a three-layer bootstrapping strategy:

\textbf{Layer 1 - Genesis attestors:} A small set of widely trusted entities provide initial attestations. These could be established community members, respected organizations, or protocol founders. Their influence naturally decays over time through trust propagation algorithms.

\textbf{Layer 2 - Objective credentials:} External verifiable credentials provide initial legitimacy signals. This includes on-chain history (past participation in DAOs, protocol usage), off-chain credentials (GitHub contributions, professional certifications), or token holdings above minimum thresholds.

\textbf{Layer 3 - Progressive trust building:} New members earn attestations through small contributions initially, building reputation over time. Early participation is limited but grows with earned trust.

The bootstrapping configuration becomes a critical governance parameter. Too restrictive, and the network cannot grow. Too permissive, and Sybil attacks become feasible. Successful systems will likely require iteration and adjustment as they scale. Pioneering implementations like TrustGraph demonstrate viable solutions through trusted seeds and evidence-based onboarding.

\subsection{Use Cases}

\subsubsection{Liquid Democracy with Expertise-Weighted Delegation}

Token-based delegation systems enable basic vote delegation but cannot incorporate expertise or trust. A token holder can delegate to anyone, regardless of the delegate's knowledge or alignment with their interests. This creates risks of uninformed or misaligned decision-making at scale, where popular but unqualified delegates accumulate disproportionate power.

Attestation-based liquid democracy enables sophisticated delegation markets that consider multiple dimensions of delegate suitability:

\begin{itemize}
\item \textbf{Expertise matching:} Technical proposals route to delegates with verified technical expertise through attestations from peers who have worked with them
\item \textbf{Value alignment:} Delegators can restrict delegation to those who share attested values or have demonstrated commitment to specific causes
\item \textbf{Trust boundaries:} Delegation can be limited by social distance in the attestation graph, preventing delegation to unknown actors
\item \textbf{Context-specific delegation:} Different proposal types can route to different delegates based on relevant attestations
\item \textbf{Delegation markets:} Delegates can signal their expertise areas and capacity, while delegators can discover suitable representatives through attestation-based search
\end{itemize}

Verifiable Services compute these complex delegation paths by traversing the attestation graph, checking constraints, and resolving transitive delegations. The output is a merkle tree of resolved voting weights that the on-chain contract can verify without repeating the expensive computation. This enables DAOs to implement liquid democracy at scales previously impossible due to gas costs.

\subsubsection{Merit-Based Incentive Distribution}

Traditional DAOs distribute rewards based on simple metrics such as token holdings or proposal submission. This misses the vast majority of valuable contributions: thoughtful forum discussions, code reviews, community support, relationship building, and strategic guidance. These contributions are critical for DAO success but invisible to smart contracts.

Attestation-based merit systems recognize the full spectrum of contributions:

Contributors receive attestations from peers who directly observed their work. These might attest to code quality, helpful documentation, successful project management, or community building efforts. The attestations form a contribution graph showing who provided value and who recognized it.

Verifiable Services analyze this graph to compute contribution scores. The algorithm might weight recent contributions higher, recognize diverse contribution types, factor in the credibility of attestors, and detect and discount reciprocal attestation rings. The resulting scores determine reward distributions, creating incentives for the actual work that makes DAOs successful.

\subsubsection{Cross-DAO Reputation and Coordination}

As the DAO ecosystem matures, coordination between organizations becomes essential for shared challenges, common resources, and avoiding destructive competition. Traditional coordination relies on informal relationships or simple token-based voting across organizations. Attestation networks enable inter-organizational coordination that can scale while maintaining legitimacy.

Organizations attest to relationships with other DAOs, shared values, collaboration history, and trust levels. Individual members attest to participation in multiple organizations and to expertise in cross-organizational coordination. Verifiable services analyze these networks to identify natural coalition partners, detect conflicts of interest, and recommend coordination strategies aligned with each organization's stated goals and community preferences.

The system enables federated governance where related DAOs coordinate decision-making without surrendering autonomy. For ecosystem-wide challenges such as security standards, infrastructure funding, or regulatory responses, DAOs can participate in meta-governance where voting power reflects both organizational stake and attestation-based legitimacy. This avoids domination by large treasuries while reflecting genuine community preferences across organizations.

Cross-DAO attestation networks also enable reputation portability. Contributions to one organization can be recognized by others in the same ecosystem. This reduces friction for contributors who participate in multiple DAOs and creates incentives for collaboration.

\subsection{Theoretical Considerations for Implementation}

While attestation-based governance offers significant potential, implementations must address several theoretical challenges:

\textbf{Bootstrapping and Network Effects:} Cold-start presents a fundamental challenge—initial participants have no one to attest to them, while the value of attestations depends on network density. Implementations must design bootstrapping mechanisms that balance initial accessibility with long-term Sybil resistance. Pioneering implementations like TrustGraph demonstrate viable solutions through trusted seeds and evidence-based onboarding.

\textbf{Gaming and Coordination Risks:} Public attestation systems face potential manipulation through reciprocal attestation rings or social pressure dynamics. Decay mechanisms, negative attestations, and multi-dimensional scoring can mitigate but not eliminate these risks. The challenge lies in designing mechanisms robust to adversarial behavior while remaining accessible to legitimate participants.

\textbf{Privacy and Transparency Balance:} Attestations create permanent records that may reveal sensitive relationships or evaluations. While transparency enhances accountability, it may discourage honest negative feedback or expose professional networks. Future implementations might explore zero-knowledge attestations that prove properties without revealing specifics.

\textbf{Temporal Validity and Evolution:} Trust relationships change over time, requiring mechanisms for attestation expiry, updates, and revocation. Design choices around temporal validity significantly impact both system dynamics and computational requirements.

These considerations represent active areas of exploration rather than fundamental barriers. Each implementation will make different tradeoffs based on specific use cases and community values.

\subsection{Benefits of Attestation-based Governance Architecture}

Attestation-based governance architecture offers fundamental advantages over traditional DAO systems while enabling new organizational capabilities.

\begin{itemize}
\item \textbf{Multi-Dimensional Legitimacy:} Attestation networks recognize multiple legitimacy sources simultaneously such as domain expertise, contribution history, community trust, and context-specific knowledge.
\item \textbf{Dynamic Adaptability:} New contribution types can be recognized through new schemas without costly upgrades. Relative importance of factors can evolve based on community preferences.
\item \textbf{Scalable Quality Assurance:} Expertise recognition and peer validation improve as community size increases, combining crowd wisdom with expert influence.
\item \textbf{Sybil Resistance Through Social Structure:} Attestation requirements create natural barriers to fake accounts while allowing legitimate participants to build reputation over time.
\item \textbf{Contextual Flexibility:} Different decisions can weight attestations differently, enabling nuanced governance that adapts to decision type and importance.
\item \textbf{Permissionless Innovation:} Anyone can create new attestation types and scoring algorithms, enabling governance evolution without protocol changes.
\item \textbf{Interoperability:} Attestations can be recognized across different DAOs and chains, creating portable reputation and enabling ecosystem-wide coordination.
\end{itemize}
