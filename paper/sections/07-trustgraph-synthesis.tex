\section{Synthesis: TrustGraph as Pioneering Implementation}

To demonstrate that this theoretical framework translates into practice, this section examines TrustGraph—a pioneering implementation of attestation-based governance launched in November 2024 with the Localism Fund Expert Network. This experimental pilot validates core concepts while exploring innovative extensions to the framework.

\subsection{Architecture Realized}

TrustGraph implements the three-tier architecture from Section 4.1 through concrete technical choices that demonstrate practical viability:

\textbf{Attestation Layer:} Using Ethereum Attestation Service (EAS), the network has successfully onboarded 65 experts who create peer attestations across grant-making expertise, Web3 tooling, and localism principles. The innovation of confidence weighting (0-100\%) enables nuanced trust expression—100\% for direct collaboration experience, lower values for evidence-based assessments. This granularity significantly enriches signal quality beyond binary attestations.

\textbf{Computation Layer:} WAVS (WebAssembly Verifiable Services) performs PageRank-style trust score calculations off-chain, generating scores from 34 to 1,811 (median 76). WAVS supports multiple deployment modes: distributed operator sets for economic security, TEE execution for hardware-attested integrity, or hybrid configurations combining both. This validates the framework's core premise: sophisticated governance computation becomes feasible when freed from on-chain constraints, while maintaining verifiability through operator consensus, hardware attestation, or their combination.

\textbf{Integration Layer:} TrustScores seamlessly integrate with compensation tiers (\$200-1,200 per contribution level), expert selection for grant evaluation, and planned integrations with Safe, Gardens, and Hats Protocol. This demonstrates the composability principle—attestation-based governance enhances rather than replaces existing tools.

\subsection{Bootstrapping Innovation}

TrustGraph validates and extends the bootstrapping strategy proposed in Section 4.1.5:

Starting with three trusted seeds (Monty Merlin, Benjamin Life, Patricia Parkinson), the network achieved critical mass without centralization. The PageRank algorithm's natural decay prevents permanent entrenchment while maintaining Sybil resistance through 3-hop trust propagation limits.

The crucial innovation: evidence-based onboarding for newcomers without network connections. Strong applicants submit verifiable credentials for seed operator review, receiving initial attestations based on merit. This solved the cold-start problem elegantly, enabling 12 participants to join purely on demonstrated expertise.

\subsection{Collective Intelligence Validated}

The implementation demonstrates practical collective intelligence (Section 5) through expert grant evaluation:

Experts express preferences via structured rubrics rather than simple votes, with influence weighted by TrustScore and domain-specific attestations. This creates funding recommendations that balance broad participation with recognized expertise—exactly the nuanced decision-making the framework envisions.

The deterministic PageRank computation achieves consensus on final TrustScores despite execution variations, validating the ``structured acceptance'' principle where agreement focuses on outputs rather than process.

\subsection{Key Innovations and Insights}

TrustGraph's experimentation reveals several powerful extensions to the theoretical framework:

\textbf{Trust Distribution vs. Dilution:} Attestors don't lose TrustScore when attesting others—they distribute influence across the network. This counterintuitive design encourages network growth without penalizing active participants.

\textbf{Ethical Standards as Foundation:} The network requires commitment to OpenCivics principles: good faith, honesty, feedback loops, efficacy, and inclusive listening. This social contract provides essential context that pure technical mechanisms cannot capture.

\textbf{Continuous Evolution:} Trust relationships evolve through attestation updates and revocations, creating a living graph that responds to demonstrated performance. The network adapts without governance overhead.

\subsection{Quantitative Success Metrics}

The pilot provides concrete validation of theoretical claims:

\begin{itemize}
\item \textbf{Sybil Resistance Achieved:} 3-hop decay successfully prevents gaming while maintaining network growth
\item \textbf{Computational Efficiency Confirmed:} Near-zero participant gas costs versus estimated 210M for naive on-chain implementation
\item \textbf{Capital-Free Participation:} 65 experts participate based purely on expertise, no token requirements
\item \textbf{Expertise Recognition:} Domain attestations successfully route technical decisions to qualified evaluators
\item \textbf{Inclusive Onboarding:} Evidence-based path enables participation without existing connections
\end{itemize}

\subsection{Replication Blueprint}

TrustGraph demonstrates a replicable pattern for implementing attestation-based governance:

\begin{enumerate}
\item Start with specific use case (expert selection) rather than full governance overhaul
\item Bootstrap with 3-5 trusted seeds for initial density
\item Implement merit-based onboarding alongside network effects
\item Add confidence weighting for signal quality
\item Integrate incrementally with existing tools
\item Establish social contract alongside technical mechanisms
\end{enumerate}

This synthesis proves the framework is not merely academically sound but practically achievable. A days-old pilot successfully operates production governance for real capital allocation, validating the paper's core thesis: sophisticated off-chain computation unlocks governance mechanisms previously impossible within blockchain constraints.
